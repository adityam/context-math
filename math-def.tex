%D \module
%D   [       file=math-tex,
%D        version=2001.04.12,
%D       subtitle=Plain Specials,
%D         author={Hans Hagen, Taco Hoekwater \& Aditya Mahajan},
%D           date=\currentdate,
%D      copyright=\PRAGMA]
%C
%C This module is part of the \CONTEXT\ macro||package and is
%C therefore copyrighted by \PRAGMA. See mreadme.pdf for
%C details.

\unprotect

\ifx\mrfam\undefined \chardef\mrfam\plusone \fi

\startluacode
    mathematics.define()
    mathematics.register_xml_entities()
\stopluacode

% special .. todo

\mathcode`\ ="8000 \mathcode`\_="8000 \mathcode`\'="8000

% will be attributes

\setfalse \automathpunctuation

\def\enablemathpunctuation {\settrue \automathpunctuation}
\def\disablemathpunctuation{\setfalse\automathpunctuation}

\def\v!autopunctuation{autopunctuation}

\appendtoks
    \doifelse{\mathematicsparameter\v!autopunctuation}\v!yes\enablemathpunctuation\disablemathpunctuation
\to \everysetupmathematics

\appendtoks
    \ifconditional\automathpunctuation\dosetattribute{mathpunc}\plusone\fi
\to \everymathematics

\setupmathematics[\v!autopunctuation=\v!yes]

% will go to math-ext

\Umathchardef\braceld=0 \mrfam "FF07A
\Umathchardef\bracerd=0 \mrfam "FF07B
\Umathchardef\bracelu=0 \mrfam "FF07C
\Umathchardef\braceru=0 \mrfam "FF07D

% ctx specific

\def\|{|} % still letter

% The \mfunction macro is an alternative for \hbox with a
% controlable font switch.

\definemathcommand [arccos]  [nolop] {\mfunction{arccos}}
\definemathcommand [arcsin]  [nolop] {\mfunction{arcsin}}
\definemathcommand [arctan]  [nolop] {\mfunction{arctan}}
\definemathcommand [arg]     [nolop] {\mfunction{arg}}
\definemathcommand [cosh]    [nolop] {\mfunction{cosh}}
\definemathcommand [cos]     [nolop] {\mfunction{cos}}
\definemathcommand [coth]    [nolop] {\mfunction{coth}}
\definemathcommand [cot]     [nolop] {\mfunction{cot}}
\definemathcommand [csc]     [nolop] {\mfunction{csc}}
\definemathcommand [deg]     [nolop] {\mfunction{deg}}
\definemathcommand [det]     [limop] {\mfunction{det}}
\definemathcommand [dim]     [nolop] {\mfunction{dim}}
\definemathcommand [exp]     [nolop] {\mfunction{exp}}
\definemathcommand [gcd]     [limop] {\mfunction{gcd}}
\definemathcommand [hom]     [nolop] {\mfunction{hom}}
\definemathcommand [inf]     [limop] {\mfunction{inf}}
\definemathcommand [injlim]  [limop] {\mfunction{inj\,lim}}
\definemathcommand [ker]     [nolop] {\mfunction{ker}}
\definemathcommand [lg]      [nolop] {\mfunction{lg}}
\definemathcommand [liminf]  [limop] {\mfunction{lim\,inf}}
\definemathcommand [limsup]  [limop] {\mfunction{lim\,sup}}
\definemathcommand [lim]     [limop] {\mfunction{lim}}
\definemathcommand [ln]      [nolop] {\mfunction{ln}}
\definemathcommand [log]     [nolop] {\mfunction{log}}
\definemathcommand [median]  [limop] {\mfunction{median}}
\definemathcommand [max]     [limop] {\mfunction{max}}
\definemathcommand [min]     [limop] {\mfunction{min}}
\definemathcommand [mod]     [limop] {\mfunction{mod}}
\definemathcommand [div]     [limop] {\mfunction{div}}
\definemathcommand [projlim] [limop] {\mfunction{proj\,lim}}
\definemathcommand [Pr]      [limop] {\mfunction{Pr}}
\definemathcommand [sec]     [nolop] {\mfunction{sec}}
\definemathcommand [sinh]    [nolop] {\mfunction{sinh}}
\definemathcommand [sin]     [nolop] {\mfunction{sin}}
\definemathcommand [sup]     [limop] {\mfunction{sup}}
\definemathcommand [tanh]    [nolop] {\mfunction{tanh}}
\definemathcommand [tan]     [nolop] {\mfunction{tan}}

\definemathcommand [integers]        {\mfunction{Z}}
\definemathcommand [reals]           {\mfunction{R}}
\definemathcommand [rationals]       {\mfunction{Q}}
\definemathcommand [naturalnumbers]  {\mfunction{N}}
\definemathcommand [complexes]       {\mfunction{C}}
\definemathcommand [primes]          {\mfunction{P}}

% using attributes

\def\choosemathbig#1#2{\dosetattribute{mathsize}{#1}\left#2\right.\doresetattribute{mathsize}}

\definemathcommand [big]  {\choosemathbig\plusone  }
\definemathcommand [Big]  {\choosemathbig\plustwo  }
\definemathcommand [bigg] {\choosemathbig\plusthree}
\definemathcommand [Bigg] {\choosemathbig\plusfour }

\definemathcommand [bigl]  [open]  [one] {\big}
\definemathcommand [bigm]  [rel]   [one] {\big}
\definemathcommand [bigr]  [close] [one] {\big}
\definemathcommand [Bigl]  [open]  [one] {\Big}
\definemathcommand [Bigm]  [rel]   [one] {\Big}
\definemathcommand [Bigr]  [close] [one] {\Big}
\definemathcommand [biggl] [open]  [one] {\bigg}
\definemathcommand [biggm] [rel]   [one] {\bigg}
\definemathcommand [biggr] [close] [one] {\bigg}
\definemathcommand [Biggl] [open]  [one] {\Bigg}
\definemathcommand [Biggm] [rel]   [one] {\Bigg}
\definemathcommand [Biggr] [close] [one] {\Bigg}

% special

%AM: Optimize this! Add similar options for sums.

\def\setoperatorlimits#1#2% operator limits
  {\savenormalmeaning{#1}%
   \def#1{\getvalue{normal\strippedcsname#1}#2}}

\setoperatorlimits\int              \intlimits
\setoperatorlimits\iint             \intlimits
\setoperatorlimits\iiint            \intlimits
\setoperatorlimits\oint             \intlimits
\setoperatorlimits\oiint            \intlimits
\setoperatorlimits\oiiint           \intlimits
\setoperatorlimits\intclockwise     \intlimits
\setoperatorlimits\ointclockwise    \intlimits
\setoperatorlimits\ointctrclockwise \intlimits

% todo: virtual in math-vfu

% \definemathcommand [mapsto]         {\mapstochar\rightarrow}
% \definemathcommand [hookrightarrow] {\lhook\joinrel\rightarrow}
% \definemathcommand [hookleftarrow]  {\leftarrow\joinrel\rhook}
% \definemathcommand [bowtie]         {\mathrel\triangleright\joinrel\mathrel\triangleleft}
% \definemathcommand [models]         {\mathrel|\joinrel=}
% \definemathcommand [iff]            {\;\Longleftrightarrow\;}

% hm

% ldots = 2026
% vdots = 22EE
% cdots = 22EF
% ddots = 22F1
% udots = 22F0

% \def\PLAINldots{\ldotp\ldotp\ldotp}
% \def\PLAINcdots{\cdotp\cdotp\cdotp}

% \def\PLAINvdots
%   {\vbox{\baselineskip.4\bodyfontsize\lineskiplimit\zeropoint\kern.6\bodyfontsize\hbox{.}\hbox{.}\hbox{.}}}

% \def\PLAINddots
%   {\mkern1mu%
%    \raise.7\bodyfontsize\vbox{\kern.7\bodyfontsize\hbox{.}}%
%    \mkern2mu%
%    \raise.4\bodyfontsize\relax\hbox{.}%
%    \mkern2mu%
%    \raise.1\bodyfontsize\hbox{.}%
%    \mkern1mu}

% \definemathcommand [ldots] [inner]   {\PLAINldots}
% \definemathcommand [cdots] [inner]   {\PLAINcdots}
% \definemathcommand [vdots] [nothing] {\PLAINvdots}
% \definemathcommand [ddots] [inner]   {\PLAINddots}

%D \starttyping
%D $\sqrt[3]{10}$
%D \stoptyping

\def\rootradical{\Uroot 0 "221A } % can be done in char-def

\def\root#1\of{\rootradical{#1}} % #2

\unexpanded\def\sqrt{\doifnextoptionalelse\rootwithdegree\rootwithoutdegree}

\def\rootwithdegree   [#1]{\rootradical{#1}}
\def\rootwithoutdegree    {\rootradical  {}}

\def\PLAINmatrix#1%
  {\null\,\vcenter{\normalbaselines\mathsurround\zeropoint
   \ialign{\hfil$##$\hfil&&\quad\hfil$##$\hfil\crcr
   \mathstrut\crcr\noalign{\kern-\baselineskip}
   #1\crcr\mathstrut\crcr\noalign{\kern-\baselineskip}}}\,}

\definemathcommand [mathstrut] {\vphantom{(}}
\definemathcommand [joinrel]   {\mathrel{\mkern-3mu}}

% \definemathcommand [matrix]    {\PLAINmatrix}
% \definemathcommand [over]      {\normalover} % hack, to do

\unexpanded\def\{{\mathortext\lbrace\letterleftbrace }
\unexpanded\def\}{\mathortext\rbrace\letterrightbrace}

%D The following colon related definitions are provided by Aditya
%D Mahajan who derived them from \type {mathtools.sty} and \type
%D {colonequals.sty}.

%D \macros
%D   {centercolon, colonminus, minuscolon, colonequals, equalscolon,
%D    colonapprox, approxcolon, colonsim, simcolon, coloncolon,
%D    coloncolonminus, minuscoloncolon, coloncolonequals,
%D    equalscoloncolon, coloncolonapprox, approxcoloncolon,
%D    colonsim, simcoloncolon}
%D
%D In $a := b$ the colon is not vertically centered with the equal
%D to. Also the distance between colon and equal to is a bit large.
%D So, we define a vertically centered colon \tex {centercolon} and
%D a few macros for colon and double colon relation symbols.
%D
%D \startlines
%D \formula {A \centercolon       B}
%D \formula {A \colonminus        B}
%D \formula {A \minuscolon        B}
%D \formula {A \colonequals       B}
%D \formula {A \equalscolon       B}
%D \formula {A \colonapprox       B}
%D \formula {A \approxcolon       B}
%D \formula {A \colonsim          B}
%D \formula {A \simcolon          B}
%D \formula {A \coloncolon        B}
%D \formula {A \coloncolonminus   B}
%D \formula {A \minuscoloncolon   B}
%D \formula {A \coloncolonequals  B}
%D \formula {A \equalscoloncolon  B}
%D \formula {A \coloncolonapprox  B}
%D \formula {A \approxcoloncolon  B}
%D \formula {A \colonsim          B}
%D \formula {A \simcoloncolon     B}
%D \stoplines

%D The next macros take care of the space between the colon and the
%D relation symbol.

\definemathcommand  [colonsep]        {\mkern-1.2mu}
\definemathcommand  [doublecolonsep]  {\mkern-0.9mu}

%D The next macro vertically centeres its contents.

\def\@center@math#1%
  {\vcenter{\hbox{$\mathsurround\zeropoint#1$}}}

\def\@center@colon
  {\mathpalette\@center@math{\colon}}

%D Now we define all the colon relations.

\definemathcommand [centercolon]      [rel] {\@center@colon}
\definemathcommand [colonminus]       [rel] {\centercolon\colonsep\mathrel{-}}
\definemathcommand [minuscolon]       [rel] {\mathrel{-}\colonsep\centercolon}
\definemathcommand [colonequals]      [rel] {\centercolon\colonsep=}
\definemathcommand [equalscolon]      [rel] {=\centercolon\colonsep}
\definemathcommand [colonapprox]      [rel] {\centercolon\colonsep\approx}
\definemathcommand [approxcolon]      [rel] {\approx\centercolon\colonsep}
\definemathcommand [colonsim]         [rel] {\centercolon\colonsep\sim}
\definemathcommand [simcolon]         [rel] {\sim\centercolon\colonsep}

\definemathcommand [coloncolon]       [rel] {\centercolon\doublecolonsep\centercolon}
\definemathcommand [coloncolonminus]  [rel] {\coloncolon\colonsep\mathrel{-}}
\definemathcommand [minuscoloncolon]  [rel] {\mathrel{-}\colonsep\coloncolon}
\definemathcommand [coloncolonequals] [rel] {\coloncolon\colonsep=}
\definemathcommand [equalscoloncolon] [rel] {=\coloncolon\colonsep}
\definemathcommand [coloncolonapprox] [rel] {\coloncolon\colonsep\approx}
\definemathcommand [approxcoloncolon] [rel] {\approx\coloncolon\colonsep}
\definemathcommand [colonsim]         [rel] {\coloncolon\colonsep\sim}
\definemathcommand [simcoloncolon]    [rel] {\sim\coloncolon\colonsep}

%D Goodies. We might move this elsewhere.

\def\underleftarrow #1{\mathop{\Uunderdelimiter 0 "2190 {#1}}}
\def\overleftarrow  #1{\mathop{\Uoverdelimiter  0 "2190 {#1}}}
\def\underrightarrow#1{\mathop{\Uunderdelimiter 0 "2192 {#1}}}
\def\overrightarrow #1{\mathop{\Uoverdelimiter  0 "2192 {#1}}}

% todo: \Udelimiterover, \Udelimiterunder

\def\normaldoublebrace {\Umathaccents 0 0 "23DE 0 0 "23DF }
\def\normaldoubleparent{\Umathaccents 0 0 "23DC 0 0 "23DD }

\let\normaloverbrace      \overbrace
\let\normalunderbrace     \underbrace
\let\normaloverparent     \overparent
\let\normalunderparent    \underparent
\let\normalunderleftarrow \underleftarrow
\let\normaloverleftarrow  \overleftarrow
\let\normalunderrightarrow\underrightarrow
\let\normaloverrightarrow \overrightarrow

\unexpanded\def\mathopwithlimits#1#2{\mathop{#1{#2}}\limits}
\unexpanded\def\stackrel        #1#2{\mathrel{\mathop{#2}\limits^{#1}}}

\unexpanded\def\overbrace      {\mathopwithlimits\normaloverbrace      }
\unexpanded\def\underbrace     {\mathopwithlimits\normalunderbrace     }
\unexpanded\def\doublebrace    {\mathopwithlimits\normaldoublebrace    }
\unexpanded\def\overparent     {\mathopwithlimits\normaloverparent     }
\unexpanded\def\underparent    {\mathopwithlimits\normalunderparent    }
\unexpanded\def\doubleparent   {\mathopwithlimits\normaldoubleparent   }
\unexpanded\def\underleftarrow {\mathopwithlimits\normalunderleftarrow }
\unexpanded\def\overleftarrow  {\mathopwithlimits\normaloverleftarrow  }
\unexpanded\def\underrightarrow{\mathopwithlimits\normalunderrightarrow}
\unexpanded\def\overrightarrow {\mathopwithlimits\normaloverrightarrow }

% todo mathclass=punctuation ord

% \Umathcode"02C="6 "0 "02C
% \Umathcode"02E="0 "0 "02E

% tricky .. todo

\appendtoks
    \def\over{\primitive\over}%
\to \everymathematics

\protect \endinput
